\matrix [list=a] {
  \node [index=1]; &
  \node [index=2]; &
  \node [index=3]; &
  \node [elements between]; &
  \node [index=i]; &
  \node [elements between]; &
  \node [index=n]; \\
};

\matrix [list=b, below=2 of list a] {
  \node (b1) {$a_1$}; &
  \node [index=2]; &
  \node [index=3]; &
  \node [elements between]; &
  \node [index=i]; &
  \node [elements between]; &
  \node [index=n]; \\
};

\tikzset{
  scan fun/.style={
    fun,
    function arity=2,
  }
}

\foreach \i in {2,3,i,n}{
  \node (fun\i) at ($ (a\i)!.5!(b\i) $) [scan fun];
  \draw [<-] (fun\i.in 1) -- (\currcoord |- a\i.south);
  \draw [->] (fun\i.out) -- (b\i.north);
}

\coordinate (top rule) at ($ (a2.south)!.5!(fun2.north) $);
\coordinate (bottom rule) at ($ (b2.north)!.5!(fun2.south) $);

\draw [->] (a1.south) -- (b1.north);
\draw [->] (top rule -| a1.south) -| (fun2.in 2);

\coordinate (x) at ($ (fun2)!.5!(fun3) $);
\draw [->] (fun2.out |- bottom rule) -| (x) -- (\currcoord |- top rule) -| (fun3.in 2);

\coordinate (x) at ($ (fun3)!.5!(funi) $);
\draw [->, dashed] (fun3.out |- bottom rule) -| (x) -- (\currcoord |- top rule) -| (funi.in 2);

\coordinate (x) at ($ (funi)!.5!(funn) $);
\draw [->, dashed] (funi.out |- bottom rule) -| (x) -- (\currcoord |- top rule) -| (funn.in 2);

