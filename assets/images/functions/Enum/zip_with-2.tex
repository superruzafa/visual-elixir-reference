\def\ziplistitemsize{1cm}
\matrix [list=z1, element size=\ziplistitemsize] {
  \node (z1a) {$a_1$}; &
  \node (z1b) {$b_1$}; &
  \node (z1c) {$c_1$}; &
  \node [elements between=.5]; &
  \node (z1k) {$k_1$}; \\
};

\matrix [list=z2, element size=\ziplistitemsize, right=.5 of list z1] {
  \node (z2a) {$a_2$}; &
  \node (z2b) {$b_2$}; &
  \node (z2c) {$c_2$}; &
  \node [elements between=.5]; &
  \node (z2k) {$k_2$}; \\
};

\matrix [list=zn, element size=\ziplistitemsize, right=1 of list z2] {
  \node (zna) {$a_n$}; &
  \node (znb) {$b_n$}; &
  \node (znc) {$c_n$}; &
  \node [elements between=.5]; &
  \node (znk) {$k_n$}; \\
};


\matrix [list=e, wrapper, element size=\ziplistitemsize, above=1 of list z2] {
  \node (e1) [placeholder=4]; &
  \node (e2) [placeholder=3.5]; &
  \node (e3) [placeholder=3.5]; &
  \node [elements between=1]; &
  \node (e4) [placeholder=4]; \\
};


\matrix at (e1) [list=a, element size=\ziplistitemsize] {
  \node [index=1]; &
  \node [index=2]; &
  \node [elements between=.5]; &
  \node [index=n]; &
  \node [elements after=.5]; \\
};

\matrix at (e2) [list=b, element size=\ziplistitemsize] {
  \node [index=1]; &
  \node [index=2]; &
  \node [elements between=.5]; &
  \node [index=n]; \\
};

\matrix at (e3) [list=c, element size=\ziplistitemsize] {
  \node [index=1]; &
  \node [index=2]; &
  \node [elements between=.5]; &
  \node [index=n]; \\
};

\matrix at (e4) [list=k, element size=\ziplistitemsize] {
  \node [index=1]; &
  \node [index=2]; &
  \node [elements between=.5]; &
  \node [index=n]; &
  \node [elements after=.5]; \\
};

%\foreach \e in {a,b,c,k}{
  %\foreach \i in {1,2,n}{
    %\draw [->, out=270, in=90] (\e\i.south) to (z\i\e.north);
  %}
%}

\node at ($ (list e.south)!.5!(list z2.north) $) {$\Downarrow$};

\tikzset{
  zip_fun/.style={
    fun={\texttt{fun}},
  }
}

\foreach \i in {1,2,n}{
  \node (zip_fun \i) [zip_fun, below=1 of list z\i];
  \draw [brace] (list z\i.south east) -- (list z\i.south west);
  \draw [->] (last brace) -- (zip_fun \i.in);
};

\coordinate (x) at ($ (zip_fun 1.south west)!.5!(zip_fun n.south east) $);

\matrix [list=z, below=1 of x] {
  \node [index=1]; &
  \node [index=2]; &
  \node [elements between=.5]; &
  \node [index=n]; \\
};

\foreach \i in {1,2,n}{
  \draw [->, out=270, in=90] (zip_fun \i.out) to (z\i.north);
};

