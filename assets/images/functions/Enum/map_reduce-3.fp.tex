\tikzset{
  fun 2/.style={
    fun={\texttt{fun}},
    function arity=2,
  }
}

\matrix [list=a] {
  \node [index=1]; &
  \node [index=2]; &
  \node [elements between]; &
  \node [index=n]; \\
};

\foreach \i in {1,2,n}{
  \draw [->] (a\i.south) -- +(0, -1.2)
    node (fun \i) [fun 2, below, anchor=in 1];
}

\coordinate (rule) at ($ (a1.south)!.5!(fun 1.in 1) $);

\draw [<-] (fun 1.in 2) |- (\currcoord |- rule) -- +(-1, 0)
  node [left] {\texttt{acc}};

\begin{scope}
  [
    start chain=tuple going base right,
    every node/.style={
      on chain=tuple,
      inner xsep=0,
      inner ysep=1mm,
    }, 
    node distance=0
  ]
  \node (a) [below left=.75 and .375 of fun 1.out] {$\{$};
  \node (b1_) {$b_1$};
  \node {,\,};
  \node (acc1) {$acc_1$};
  \node (b) {$\}$};
\end{scope}

\node (x) [fit=(a) (b)] {};
\draw [->] (fun 1.out) -- (\currcoord |- x.north);

\coordinate (x) at ($ (fun 1)!.5!(fun 2) $);
\draw [->] (acc1.north -| x) |- (\currcoord |- rule) -| (fun 2.in 2);




\begin{scope}
  [
    start chain=tuple going base right,
    every node/.style={
      on chain=tuple,
      inner xsep=0,
      inner ysep=1mm,
    }, 
    node distance=0
  ]
  \node (a) [below left=1.5 and .25 of fun 2.out] {$\{$};
  \node (b2_) {$b_2$};
  \node {,\,};
  \node (acc2) {$acc_2$};
  \node (b) {$\}$};
\end{scope}

\node (x) [fit=(a) (b)] {};
\draw [->] (fun 2.out) -- (\currcoord |- x.north);

%\coordinate (x) at ($ (fun 2)!.5!(fun i) $);
\draw [->, dashed] (acc2.north) |- (\currcoord |- rule) -| (fun n.in 2);



\begin{scope}
  [
    start chain=tuple going base right,
    every node/.style={
      on chain=tuple,
      inner xsep=0,
      inner ysep=1mm,
    }, 
    node distance=0
  ]
  \node (a) [below left=2.25 and .25 of fun n.out] {$\{$};
  \node (bn_) {$b_n$};
  \node {,\,};
  \node (accn_) {$acc_n$};
  \node (b) {$\}$};
\end{scope}

\node (x) [fit=(a) (b)] {};
\draw [->] (fun n.out) -- (\currcoord |- x.north);

\matrix [tuple=t, wrapper, element size=2cm, below=5 of list a] {
  \node (t1) [placeholder=3.5]; & \node [separator]; &
  \node (accn) {$acc_n$}; \\
};

\matrix at (t1) [list=b] {
  \node [index=1]; &
  \node [index=2]; &
  \node [elements between=.5]; &
  \node [index=n]; \\
};

\foreach \i in {1,2,n}{
  \draw [->, out=270, in=90] (b\i_.south) to (b\i.north);
}

\draw [->, out=270, in=90] (accn_.south) to (accn.north);

%\coordinate (rule) at ($ (a.south)!.5!(fun1.north) $);

%\path let
    %\p1=(fun1.east),
    %\p2=(fun2.west),
    %\n1={(\x2 - \x1) / 2}
%in
    %\pgfextra{
        %\global\edef\offset{\n1}
    %}
%;
%\draw [fptk, <- flow] (fun1.north io 2) -- (\currcoord |- rule) -- +(-1, 0)
    %node [left] {\texttt{acc}};

%\foreach \i/\j in {1/.5,2/1.5,i/.5,n/.5}{
    %\coordinate (x) at ($ (fun\i.east) + (\offset, 0) $);
    %\edef\tacc{\txacc[\i]}
    %\matrix [fptk, subtuple=t\i] [
        %below=of {{$ (fun\i.south io) + (0, -\j) $} -| x},
        %matrix anchor=\tacc.north,
        %ampersand replacement=\&,
    %] {
        %\elemx{b}{$b_\i$} \& \comma \&
        %\elemx{acc}{$acc_\i$} \\
    %};
    %\draw [fptk, subflow ->] (fun\i.south io) -- (\currcoord |- t\i.north);
%}

%\draw [fptk, subflow ->, flow shape |-|=rule] (t1acc.north) -- (fun2.north io 2);
%\draw [fptk, subflow ->, flow shape |..|=rule] (t2acc.north) -- (funi.north io 2);
%\draw [fptk, subflow ->, flow shape |..|=rule] (tiacc.north) -- (funn.north io 2);

%\matrix[fptk, list=m, element=b, below=4.5 of a] {
    %\node [index=1]; &
    %\node [index=2]; &
    %\node [elements between]; &
    %\node [index=i]; &
    %\node [elements between]; &
    %\node [index=n]; \\
%};

%\foreach \i in {1,2,i,n}{
    %\draw [fptk, flow ->] (\txb[\i]) -- (\currcoord |- m\i.north);
%}

%\matrix at (m.base) [fptk, tuplex={T}{1.25}, matrix anchor=T1.base] {
    %\placeholder[1]{7} & \comma &
    %\elemx{acc}{$acc_n$} \\
%};

%\draw [fptk, flow ->=soft] (tnacc.south) -- (Tacc.north);
