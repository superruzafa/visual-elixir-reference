\makeatletter

\newif\iffptk@range@enabled
\def\fptk@range@segments{1/solid}

\def\fptk@range@arrow@direct{0}
\def\fptk@range@arrow@inverse{1}
\def\fptk@range@arrow@none{2}
\let\fptk@range@arrow\fptk@range@arrow@direct

\pgfkeys{
    /fptk/.cd,
    range segments/.code={
        \def\fptk@range@segments{#1}
        \iffptk@range@enabled
            \pgfkeysalso{/fptk/range}
        \fi
    },
    range arrow/.is choice,
    range arrow/direct/.code=\let\fptk@range@arrow\fptk@range@arrow@direct,
    range arrow/inverse/.code=\let\fptk@range@arrow\fptk@range@arrow@inverse,
    range arrow/none/.code=\let\fptk@range@arrow\fptk@range@arrow@none,
    range start label/.initial=,
    range end label/.initial=,
    range/.code={
        \fptk@range@enabledtrue
        \pgfkeys{
            /tikz/.cd,
            decorate,
            decoration={
                show path construction,
                lineto code={
                    \path (\tikzinputsegmentfirst);
                    \pgfgetlastxy{\fptk@temp@xa}{\fptk@temp@ya}
                    \path (\tikzinputsegmentlast);
                    \pgfgetlastxy{\fptk@temp@xb}{\fptk@temp@yb}
                    \pgfmathveclen{\fptk@temp@xb - \fptk@temp@xa}{\fptk@temp@yb - \fptk@temp@ya}
                    \edef\fptk@temp@veclen{\pgfmathresult pt}
                    \pgfmathatantwo{\fptk@temp@yb - \fptk@temp@ya}{\fptk@temp@xb - \fptk@temp@xa}
                    \let\fptk@temp@xc\pgfmathresult
                    \def\thedraw{
                        \draw let
                            \p1=(\fptk@temp@xa),
                            \p3=(\x1, \y1 - 3mm),
                            \n1={1cm}
                        in
                            [-{Stealth[inset=0pt, length=3mm, angle'=30]}]
                            (\p3) -- +(\fptk@temp@xb, 0)
                        ;
                    }
                    \begin{scope}
                        [transform canvas={rotate around={\fptk@temp@xc:(\tikzinputsegmentfirst)}}]
                        \foreach \i/\style [remember=\i as \lasti (initially 0)] in \fptk@range@segments {
                            \draw
                                [
                                    \style,
                                    line width=\pgflinewidth,
                                    |-|,
                                    shorten <=-.5\pgflinewidth,
                                    shorten >=-.5\pgflinewidth
                                ]
                                ($ (\tikzinputsegmentfirst) + (\lasti * \fptk@temp@veclen , 0) $) -- ($ (\tikzinputsegmentfirst) + (\i * \fptk@temp@veclen, 0) $)
                            ;
                        }
                        \ifnum\fptk@range@arrow=\fptk@range@arrow@direct
                            \def\fptk@temp@xa{\tikzinputsegmentfirst}
                            \def\fptk@temp@xb{1cm}
                            \thedraw
                        \else
                            \ifnum\fptk@range@arrow=\fptk@range@arrow@inverse
                                \def\fptk@temp@xa{\tikzinputsegmentlast}
                                \def\fptk@temp@xb{-1cm}
                                \thedraw
                            \fi
                        \fi
                    \end{scope}
                    \node [above=4mm of \tikzinputsegmentfirst, anchor=base] {\pgfkeysvalueof{/tikz/fptk/range start label}};
                    \node [above=4mm of \tikzinputsegmentlast, anchor=base] {\pgfkeysvalueof{/tikz/fptk/range end label}};
                }
            }
        }
    }
}

\newcommand\rangetop[5]{
    \draw [line width=3\defaultpenwidth] [
        fptk,
        range,
        range start label=#4,
        range end label=#5,
        range segments={#3},
    ] ([yshift=.5\masterunit] #1) -- ([yshift=.5\masterunit] #2);
}

\newcommand\rangeinversetop[5]{
    \draw [line width=3\defaultpenwidth] [
        fptk,
        range,
        range start label=#4,
        range end label=#5,
        range segments={#3},
        range arrow=inverse,
    ] ([yshift=.5\masterunit] #1) -- ([yshift=.5\masterunit] #2);
}

\makeatother